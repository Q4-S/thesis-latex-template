% ============================================
% CUSTOMIZATION SECTION - EDIT THESE SETTINGS & REQUIRED PACKAGES
% ============================================
\usepackage{fontspec}			% use compiler XeLaTeX/LuaLaTeX
\usepackage{kotex} 				% kotex to use Korean/Hangeul font, alternative: xeCJK
% Using 'xeCJK' need to change hangulfont set-command. It is better if you learn manually how to install other fonts if needed. You can manually upload the font files (.ttf or .otf)

% See how Fonts worked in Apendix, not all Korean font support bold, italic or bold italic
% === MAIN FONTS (Default body text, paragraphs, normal content) ===
\setmainfont{Times New Roman}        % English in Times New Roman
\setmainhangulfont{UnBatang}[        % Korean main font
  %BoldFont = NanumMyeongjo,         % UnBatang only support bold by default
  ItalicFont = NanumMyeongjo,        % You can use other font for italic
  BoldItalicFont = NanumMyeongjo     % Or use same font for bold italic
]

% === SANS-SERIF FONTS (For Headers, titles, captions ), How to use : \textsf{} ===
\setsansfont{Arial}                 % English sans-serif
\setsanshangulfont{NanumGothic}   	% Korean sans-serif

% === MONOSPACE FONTS (For code, URLs, technical text), How to use : \texttt{}  ===
\setmonofont{Courier New}           	% English monospace
\setmonohangulfont{NanumGothicCoding}   % Korean monospace

% ================== SETTING MARGIN =======================
\usepackage[                % you must remove predefined paper size from "documentclass" if activated custom papersize
	%paperwidth=21cm,       % Custom paper width
	%paperheight=29.7cm,    % Custom paper height
	top=2.5cm,
	bottom=2.5cm,
	left=3.0cm,
	right=3.0cm,
	bindingoffset=0.5cm 	% Needed for printing document
]{geometry}

% Line Spacing
\usepackage{setspace}
\singlespacing% options: \singlespacing, \onehalfspacing, \doublespacing, \setstretch{1.25} % custom-spacing

\setlength{\parskip}{0.5cm} 	% Space between paragraphs
\setlength{\parindent}{0.7cm} 	% Paragraph indentation

% Always indent after chapter, section, or subsection See Chap. 1
\usepackage{indentfirst} 		% You can deactivate this package or use \noindent only in each first paragraph

% ================== OTHER PACKAGES =======================
%\usepackage{graphicx}           % Used to include images and other graphics in your document.
\usepackage[demo]{graphicx}

\usepackage{amsmath, amssymb}   % Essential for typesetting mathematical formulas and equations and Provides a wide range of additional math symbols.

\usepackage{hyperref}           % adds clickable links and navigation features to your PDF.
\usepackage{fancyhdr}           % Extensive control of page headers and footers
\usepackage{tocloft}			% Some setting may conflict to "titlesec" package
\usepackage[compact]{titlesec}  % Add [compact] option

%============= SETTING of ToC LoT LoF Appendix==============
% Center TOC title
\renewcommand{\cfttoctitlefont}{\hfill\Huge\bfseries}
\renewcommand{\cftaftertoctitle}{\hfill}

% Center LOF title
\renewcommand{\cftloftitlefont}{\hfill\Huge\bfseries}
\renewcommand{\cftafterloftitle}{\hfill}

% Center LOT title
\renewcommand{\cftlottitlefont}{\hfill\Huge\bfseries}
\renewcommand{\cftafterlottitle}{\hfill}

% Customize Table of Content (TOC) appearance
\renewcommand{\cftchapfont}{\bfseries}           % Bold chapter entries
\renewcommand{\cftchapdotsep}{\cftdotsep}        % Add dots for chapters
\setlength{\cftbeforechapskip}{5pt}              % Space between chapters
\renewcommand{\cftchappagefont}{\bfseries}       % Bold page numbers

\renewcommand{\cftchappresnum}{Bab } 	% Add "Chapter " prefix before chapter numbers in TOC
\renewcommand{\cftchapaftersnum}{.} 	% Add . after chapter number
\renewcommand{\cftsecaftersnum}{.} 	    % Add . after section number
\renewcommand{\cftsubsecaftersnum}{.} 	% Add . after subsection number
\setlength{\cftchapnumwidth}{2cm} 		% Adjust spacing to accommodate the prefix for Chapter
\setlength{\cftsecnumwidth}{1.6cm}		% Adjust spacing to accommodate the prefix for Section
\setlength{\cftsubsecnumwidth}{1.7cm}	% Adjust spacing to accommodate the prefix For sub-Section

\setlength{\cftchapindent}{0em}		% indent for Chapter
\setlength{\cftsecindent}{1em}		% indent for Section
\setlength{\cftsubsecindent}{2em}	% indent for sub-Section

% Customize List of Figures (LoF) appearance
\setlength{\cftfigindent}{0em}        % Indent from left
\setlength{\cftfignumwidth}{3em}      % Width for "Figure 1.1"
\setlength{\cftbeforefigskip}{2pt}    % Space between entries

% Customize List of Tables (LoT) appearance
\setlength{\cfttabindent}{0em}        % Indent from left
\setlength{\cfttabnumwidth}{3em}      % Width for "Table 1.1"
\setlength{\cftbeforetabskip}{2pt}    % Space between entries

% Custom name for ToC LoT LoF Appendix 
\renewcommand{\contentsname}{Daftar Isi}
\renewcommand{\listfigurename}{Daftar Gambar}
\renewcommand{\listtablename}{Daftar Tabel}
\renewcommand{\appendixname}{Apendiks}

% ================== SETTING of CHAPTER NAME & TITLES in Main Document =======================
% Centering chapter titles and make it one line
\titleformat{\chapter} % command to format
[block] % shape: hang, display, block, frame etc
{\sffamily\Huge\bfseries\centering}  % format of label + chapter title
{\chaptertitlename\ \thechapter.} % add dot label "Chapter 1."
{0.5cm} % separation label - chapter title
{} % code before

% To change format the chapter name  
\renewcommand{\chaptername}{제:}				% prefix, before number of chapter
\renewcommand{\thechapter}{\arabic{chapter}장}	% suffix, after number of chapter

% Format tittles with SANS-SERIF and dot suffix for section and subsection
\titleformat{\section}{\sffamily\large\bfseries}{\thesection.}{1em}{}
\titleformat{\subsection}{\sffamily\normalsize\bfseries}{\thesubsection.}{1em}{}

% Spacing for chapters and sections
\titlespacing*{\chapter}{0cm}{0cm}{0.5cm}
\titlespacing*{\section}{0cm}{1cm}{0.5cm}
\titlespacing*{\subsection}{0cm}{1cm}{0.5cm}

%\titlespacing{\chapter or \section or \subsection} explanation:
%{1st} % left of the label + title
%{2nd} % vertical space before the title
%{3rd} % vertical space after title

% ================== SETTING of Hyperlink =======================
\hypersetup{
colorlinks=true,	  % no colored links box (for printing)
linkcolor=blue,       % Internal links (TOC, references) default is black
citecolor=blue,       % Citations for References when use \cite
urlcolor=blue         % Web URLs
}

% ================== PAGE NUMBER FORMATTING =======================

% Roman numerals for front matter
\newcommand{\frontmatterpagenumbers}{
	\pagenumbering{roman}
	\setcounter{page}{1}
}

% Arabic numerals for main content
\newcommand{\mainmatterpagenumbers}{
	\clearpage
	\pagenumbering{arabic}
	\setcounter{page}{1}
}

% Header and footer style, also for pagenumber. Learn manually by yourself if you want custom it :)

% Plain style for chapter pages
\fancypagestyle{plain}{        % Requires fancyhdr package
	\fancyhf{}                 % Clear all            
	\fancyfoot[C]{\thepage}    % centering page number
    %\fancyhead[LE,RO]{\thepage}  % Page numbers outer edge
    %\fancyhead[LO]{\leftmark}    % Chapter name left on odd
    %\fancyhead[RE]{\rightmark}   % Section name right on even
	\renewcommand{\headrulewidth}{0pt}   % For chapter pages
	\renewcommand{\footrulewidth}{0pt}   % For chapter pages
}

% Configure fancy settings
\fancyhf{}
%\fancyhead[C]{\thepage}
%\fancyhead[LE,RO]{\thepage}
\fancyfoot[C]{\thepage}
\renewcommand{\headrulewidth}{0pt}   % For normal pages
\renewcommand{\footrulewidth}{0pt}   % For normal pages

% Apply fancy style for normal pages from  Built-in LaTeX function
\pagestyle{fancy}

%\pagestyle{empty}     % Use empty style (no headers/footers)
%\pagestyle{plain}     % Use plain style (page numbers only)
%\pagestyle{headings}  % Use headings style (chapter names in header)

% ========== Optional TOC/LOF/LOT start on right page (twoside) ==========
\let\oldtableofcontents\tableofcontents
\renewcommand{\tableofcontents}{\cleardoublepage\oldtableofcontents}

\let\oldlistoffigures\listoffigures
\renewcommand{\listoffigures}{\cleardoublepage\oldlistoffigures}

\let\oldlistoftables\listoftables
\renewcommand{\listoftables}{\cleardoublepage\oldlistoftables}

% ================== THESIS INFORMATION VARIABLES =======================
\newcommand{\thesistitle}{My Thesis Title Here \\ (한국어)}
\newcommand{\thesisauthor}{My Name}
\newcommand{\supervisor}{My Professor}
\newcommand{\university}{My University Name}
\newcommand{\department}{My Department of Your Field}
\newcommand{\degree}{Doctor of Philosophy}
\newcommand{\submissiondate}{Month Year}
\newcommand{\approvaldate}{Date-Month-Year}